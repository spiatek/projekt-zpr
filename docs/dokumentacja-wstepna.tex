\documentclass[12pt,a4paper,titlepage]{article}
\usepackage[utf8x]{inputenc}
\usepackage[T1]{fontenc}
\usepackage{ucs}
\usepackage{amsmath}
\usepackage{amsfonts}
\usepackage{amssymb}
\usepackage[polish]{babel}
\frenchspacing
\sloppy
\usepackage{geometry}
\title{Zaawansowane Programowanie w C++ Dokumentacja wstępna projektu}
\author{Mateusz Matuszewski, Szymon Piątek}
\bibliographystyle{plain}
\usepackage{geometry}
\geometry{verbose,a4paper,tmargin=2cm,bmargin=2cm,lmargin=2.5cm,rmargin=2.5cm}
\begin{document}
\maketitle
\section{Temat projektu}
Stworzyć bibliotekę ułatwiającą rysowanie, porównywanie i obliczenia (np. auc) związane z krzywymi ROC, PRC. W plikach wejściowych w każdym wierszu mamy współrzędne punktu krzywej. Kolejne punkty można łączyć odcinkami. Program powinien umożliwiać wczytanie z pliku i wyświetlenie na jednym wykresie 2D kilku krzywych (każdą innym kolorem). Powinna istnieć możliwość powiększenia wybranego fragmentu krzywej. Dla wybranej krzywej program powinien podać pole powierzchni pod nią wyrażone w \% (na każdej osi maksymalna wartość to 1, zatem maksymalne pole pod krzywą wynosi 1 i to jest 100\%).
\section{Rozwinięcie tematu}
Celem tego projektu jest stworzenie aplikacji umożliwiającej porównanie krzywych ROC i PR, które są wykorzystywane do oceny jakości binarnego klasyfikatora, zaś pole pod wykresem (AUC - \textit{ang. Area Under Curve}) może zostać wykorzystane jako heurystyka przy poszukiwaniu optymalnych parametrów.

Zdefiniujmy próbki "dodatnie" (\textit{positive}) jako poszukiwaną przez nas klasę, zaś "ujemne" (\textit{negative}) jako nieinteresującą nas drugą klasę.

Krzywe ROC (\textit{ang. Receiever Operator Characteristic}) przedstawiają zmianę stosunku próbek poprawnie sklasyfikowanych jako "dodatnie" do próbek błędnie uznanych za "dodatnie" względem zmiennego parametru - poziomu odcięcia (\textit{ang. threshold}).

Drugim rodzajem krzywych, które będzie prezentować nasz program to krzywe PR (\textit{ang. Precision-Recall}) które lepiej sprawdzają się przy prezentacji wyników dla mocno niezrównoważonych zbiorów danych. Dokładność (\textit{ang. precision}) liczona jest jako stosunek poprawnie zakwalifikowanych przypadków "dodatnich" do ilości przypadków poprawnie i niepoprawnie uznanych za "dodatnie". Współczynnik \textit{recall} liczony jest jako stosunek przypadków poprawnie zakwalifikowanych jako "dodatnie" do wszystkich istotnych dla nas danych ("pozytywnych"; czyli suma poprawnych "pozytywnych" i niepoprawnych "negatywnych"). Warte zauważenia jest, że jeśli krzywa dominuje w przestrzeni ROC tylko wtedy gdy dominuje też w przestrzeni PR.

Istotną różnicą między tymi przestrzeniami jest sposób interpolacji danych. O ile dla krzywych ROC wystarczy zastosować interpolację liniową, to dla PR jest to bardziej skomplikowane. Wynika to z tego, że dla zmiennego parametru \textit{recall} dokładność niekoniecznie zmienia się liniowo. Zgodnie z \cite{roc}, interpolację można przeprowadzić korzystając ze wzoru:\\
\begin{center}
$(\frac{TP_{A}+x}{Total Pos}, \frac{TP_{A}+x}{TP_{A}+x+FP_{A} + \frac{FP_{B}-FP_{A}}{TP_{B}-TP_{A}} \cdot x})$.
\end{center}
\section{Szczegóły organizacyjno–implementacyjne}
Program zostanie napisany w języku C++, przy wykorzystaniu biblioteki Qt do tworzenia interfejsu graficznego oraz biblioteki Qwt do rysowania wykresów.
Program tworzony będzie przy użyciu systemu kontroli wersji git oraz systemu Bugzilla do zarządzania projektem i raportowania błędów.
\section{Opis działania stworzonego programu (przypadki użycia)}
\subsection{Podstawowe funkcje}
\begin{itemize}
\item \textbf{Uruchomienie programu.} Pojawia się puste okno z menu, paskiem narzędzi i paskiem stanu.
\item \textbf{Wczytanie pliku wejściowego.} Użytkownik wybiera opcję w menu lub na pasku narzędzi oraz plik o odpowiednim rozszerzeniu. Jeśli jest to możliwe, wyświetlany jest odpowiedni wykres i odświeżana jest lista krzywych. Liczone jest pole powierzchni pod krzywą. W przeciwnym wypadku wyświetlany jest komunikat o błędzie.
\item \textbf{Wyświetlenie wykresu.} Sprawdzany jest rodzaj krzywej i w zależności od tego, czy wykres danego typu figuruje w oknie, wyświetlany jest nowy wykres wraz ze sczytaną z pliku krzywą, bądź krzywa jest wyświetlana na istniejącym wykresie. Ponieważ program obsługuje dwa rodzaje krzywych (ROC i PRC), możliwe jest wyświetlenie maksymalnie dwóch wykresów. Nazwa i kolor krzywej generowane są automatycznie.
\item \textbf{Odświeżenie listy krzywych.} Do listy krzywych dodawany jest nowy element zawierający informację o sczytanej krzywej. Lista krzywych wyświetlana jest obok wykresu. We wierszu znajdują się kolejno: checkbox określający, czy krzywa jest wyświetlana na wykresie; nazwa krzywej (nadana automatycznie lub zmodyfikowana przez użytkownika) - naciśnięcie na nazwę skutkuje wyświetleniem okna z jej właściwościami; przycisk, którego naciśnięcie skutkuje wyświetleniem palety barw; przycisk, którego naciśnięcie skutkuje usunięciem krzywej.
\item \textbf{Wyświetlenie okna właściwości wykresu.} Użytkownik wybiera opcję w menu, na pasku narzędzi, lub klika dwukrotnie na obszar wykresu. Wyświetlane jest okno, w którym można ustawić podstawowe opcje wykresu (m.in. tytuł wykresu, tytuły każdej z osi, sposób wyświetlania linii siatki).
\item \textbf{Zaznaczenie fragmentu wykresu.} Użytkownik wybiera opcję w menu lub na pasku narzędzi umożliwiającą zaznaczenie kursorem fragmentu wykresu, po czym zaznacza wybrany fragment wykresu.
\item \textbf{Powiększenie fragmentu wykresu.} Użytkownik najeżdża kursorem na zaznaczony obszar i klika go w celu powiększenia.
\item \textbf{Wyświetlenie współrzędnych po najechaniu na krzywą.} Użytkownik najeżdża kursorem na punkt krzywej na wykresie. W pasku stanu wyświetlana jest informacja o współrzędnych punktu.
\item \textbf{Wyświetlenie okna właściwości krzywej.} Użytkownik wybiera krzywą z listy i klika na jej nazwę, lub dwukrotnie klika na krzywą na wykresie. Wyświetlane jest okno z informacjami o krzywej. Niektóre z nich są modyfikowalne (nazwa, kolor), inne nie (pole powierzchni pod krzywą). Dostępna jest też opcja usunięcia krzywej.
\item \textbf{Usunięcie krzywej.} Użytkownik wybiera opcję usunięcia krzywej na liście krzywych lub w oknie właściwości krzywej. Krzywa usuwana jest zarówno z wykresu, jak i z listy krzywych.
\item \textbf{Zmiana nazwy krzywej / wykresu / osi.} Użytkownik modyfikuje nazwę w oknie właściwości, po czym naciska przycisk “Zapisz”.
\item \textbf{Zmiana koloru krzywej / tła wykresu.} Użytkownik wybiera kolor z palety barw w oknie właściwości (lub w przypadku krzywych - na liście krzywych), po czym naciska przycisk “Zapisz”.
\item \textbf{Zapis wykresu do pliku.} Użytkownik wybiera z menu lub paska narzędzi opcję zapisu wykresu do pliku, wybiera wykres do zapisania, wpisuję nazwę pliku, ustawia jego lokalizację oraz format.
\item \textbf{Zamknięcie programu.} Użytkownik wybiera opcję zamknięcia programu z menu lub poprzez zamknięcie okna.
\end{itemize}
\subsection{Dodatkowe opcje}
\begin{itemize}
\item \textbf{Ustawienie linii siatki.} Użytkownik wpisuje w oknie właściwości wykresu liczbę określającą gęstość linii siatki w określonej współrzędnej, po czym naciska przycisk “Zapisz”.
\item \textbf{Zaznaczenie krzywej na wykresie.} Użytkownik klika na krzywą na wykresie. Krzywa jest zaznaczana.
\item \textbf{Usunięcie krzywej z wykresu.} Użytkownik wybiera z menu lub paska narzędzi opcję usunięcia krzywej. Zaznaczona krzywa jest usuwana z wykresu.
\end{itemize}
\begin{thebibliography}{1}
\bibitem{roc}Jesse Davis, Mark Goadrich. The Relationship Between Precision-Recall and ROC Curves. ICML '06 Proceedings of the 23rd international conference on Machine learning. 2006.
\end{thebibliography}
\end{document}